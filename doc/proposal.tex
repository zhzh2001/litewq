\documentclass[a4paper]{article}

\usepackage{amsmath,makecell,multirow,amsfonts,siunitx,amssymb,hyperref}

%\usepackage{array}
%\newcolumntype{P}[1]{>{\centering\arraybackslash}p{#1}}

\usepackage{geometry}
\geometry{left=1in,right=1in,top=1in,bottom=1in}

\usepackage{longtable,booktabs,graphicx,float,adjustbox,xurl}
\usepackage[normalem]{ulem}
\providecommand{\tightlist}{%
  \setlength{\itemsep}{0pt}\setlength{\parskip}{0pt}}
  
\usepackage{fancyvrb,fvextra,color}
\DefineVerbatimEnvironment{Highlighting}{Verbatim}{breaklines,commandchars=\\\{\}}
% Add ',fontsize=\small' for more characters per line
\newenvironment{Shaded}{}{}
\newcommand{\AlertTok}[1]{\textcolor[rgb]{1.00,0.00,0.00}{\textbf{#1}}}
\newcommand{\AnnotationTok}[1]{\textcolor[rgb]{0.38,0.63,0.69}{\textbf{\textit{#1}}}}
\newcommand{\AttributeTok}[1]{\textcolor[rgb]{0.49,0.56,0.16}{#1}}
\newcommand{\BaseNTok}[1]{\textcolor[rgb]{0.25,0.63,0.44}{#1}}
\newcommand{\BuiltInTok}[1]{#1}
\newcommand{\CharTok}[1]{\textcolor[rgb]{0.25,0.44,0.63}{#1}}
\newcommand{\CommentTok}[1]{\textcolor[rgb]{0.38,0.63,0.69}{\textit{#1}}}
\newcommand{\CommentVarTok}[1]{\textcolor[rgb]{0.38,0.63,0.69}{\textbf{\textit{#1}}}}
\newcommand{\ConstantTok}[1]{\textcolor[rgb]{0.53,0.00,0.00}{#1}}
\newcommand{\ControlFlowTok}[1]{\textcolor[rgb]{0.00,0.44,0.13}{\textbf{#1}}}
\newcommand{\DataTypeTok}[1]{\textcolor[rgb]{0.56,0.13,0.00}{#1}}
\newcommand{\DecValTok}[1]{\textcolor[rgb]{0.25,0.63,0.44}{#1}}
\newcommand{\DocumentationTok}[1]{\textcolor[rgb]{0.73,0.13,0.13}{\textit{#1}}}
\newcommand{\ErrorTok}[1]{\textcolor[rgb]{1.00,0.00,0.00}{\textbf{#1}}}
\newcommand{\ExtensionTok}[1]{#1}
\newcommand{\FloatTok}[1]{\textcolor[rgb]{0.25,0.63,0.44}{#1}}
\newcommand{\FunctionTok}[1]{\textcolor[rgb]{0.02,0.16,0.49}{#1}}
\newcommand{\ImportTok}[1]{#1}
\newcommand{\InformationTok}[1]{\textcolor[rgb]{0.38,0.63,0.69}{\textbf{\textit{#1}}}}
\newcommand{\KeywordTok}[1]{\textcolor[rgb]{0.00,0.44,0.13}{\textbf{#1}}}
\newcommand{\NormalTok}[1]{#1}
\newcommand{\OperatorTok}[1]{\textcolor[rgb]{0.40,0.40,0.40}{#1}}
\newcommand{\OtherTok}[1]{\textcolor[rgb]{0.00,0.44,0.13}{#1}}
\newcommand{\PreprocessorTok}[1]{\textcolor[rgb]{0.74,0.48,0.00}{#1}}
\newcommand{\RegionMarkerTok}[1]{#1}
\newcommand{\SpecialCharTok}[1]{\textcolor[rgb]{0.25,0.44,0.63}{#1}}
\newcommand{\SpecialStringTok}[1]{\textcolor[rgb]{0.73,0.40,0.53}{#1}}
\newcommand{\StringTok}[1]{\textcolor[rgb]{0.25,0.44,0.63}{#1}}
\newcommand{\VariableTok}[1]{\textcolor[rgb]{0.10,0.09,0.49}{#1}}
\newcommand{\VerbatimStringTok}[1]{\textcolor[rgb]{0.25,0.44,0.63}{#1}}
\newcommand{\WarningTok}[1]{\textcolor[rgb]{0.38,0.63,0.69}{\textbf{\textit{#1}}}}

%\usepackage{chngcntr}
%\counterwithin{figure}{section}
%\counterwithin{table}{section}
%\renewcommand\thefigure{\thesection-\arabic{figure}}
%\renewcommand\thetable{\thesection-\arabic{table}}

\newcommand{\includegraphicx}[1]{\maxsizebox{\textwidth}{\textheight}{\includegraphics{#1}}}

\title{LiteWQ Proposal}
\author{Zhang Zheng, Zhu Siyuan}

\begin{document}
\maketitle

\hypertarget{introduction}{%
\section{Introduction}\label{introduction}}

\begin{quote}
You are a young gray wolf. Born in the Northern Range of Yellowstone
National Park, you learned the ways of the wolf in your family pack. Now
you are venturing out on your own --- to explore, hunt, find a mate,
establish territory, and raise your own family.
\end{quote}

WolfQuest is a wolf simulation game, developed by eduweb. It is
especially popular among teenagers. It strives to provide a realistic
and immersive simulation of wolf life in Yellowstone National Park.
WolfQuest: Anniversary Edition is the latest version of the game, a
complete remake of WolfQuest Classic (2.x). Though still in Early
Access, its graphics and gameplay are already very impressive. You can
find the game on
\href{https://store.steampowered.com/app/926990/WolfQuest_Anniversary_Edition/}{Steam}.

\begin{figure}[H]
\centering
\includegraphicx{img/wqae_steam.png}
\caption{WolfQuest AE on Steam}
\end{figure}

However, WolfQuest AE doesn't have a mobile version yet, and the
available mobile version is based on WolfQuest Classic, which is
outdated and has limited graphics. We want to create a mobile version of
WolfQuest AE for Android, called \textbf{LiteWQ}. It will be a very
simple game, with only a few features of original WolfQuest. It will be
an excellent final project for Computer Graphics course.

\begin{figure}[H]
\centering
\includegraphicx{img/mobile.jpg}
\caption{WolfQuest Classic on Android}
\end{figure}

Note the quality of shadows is a bit low. We will try to improve it.

\hypertarget{features}{%
\section{Features}\label{features}}

LiteWQ is planned to have the following features:

\begin{itemize}
\tightlist
\item
  Build the game engine from scratch, using OpenGL ES 3.0
\item
  Simple 3D model loader
\item
  Render and edit materials and textures
\item
  Simple lighting system
\item
  Navigate the world using a first-person camera
\item
  Gameplay limited to scent view and tracking
\item
  Real-time collision detection (optional)
\item
  Advanced lighting system, including shadows, global illumination, and
  reflections (optional)
\end{itemize}

\begin{figure}[H]
\centering
\includegraphicx{img/scent.png}
\caption{Scent View in WQ AE}
\end{figure}

In the original game, scent view allows players to track the scent of
preys, predators, and various objects. It is an indispensable part of
the game, so we want to implement it in LiteWQ. The scents are
represented by floating particles in the air, which drift with the wind.
Both animals and their tracks can emit scent particles, so the player
can track the animals by following the scent particles. Different colors
and shapes of particles represent different animals. Other special
symbols represent carcasses, dens, scent posts (used to mark territory),
and other objects. In LiteWQ, we will probably only implement limited
subset of these features such as tracking carcasses and dens.

\begin{figure}[H]
\centering
\includegraphicx{img/animal_track.png}
\caption{Scent View (Continued)}
\end{figure}

In the original game, entering the scent view will cause the world to be
rendered in grayscale, so that the player can focus on the scent
particles. We will also implement this feature in LiteWQ. However,
normal view will not likely to be implemented, because animal animation
is difficult.

\begin{figure}[H]
\centering
\includegraphicx{img/normal.png}
\caption{Normal View in WQ AE}
\end{figure}

\hypertarget{game-engine}{%
\section{Game Engine}\label{game-engine}}

OpenGL ES 3.0 is a subset of OpenGL 3.3, which is the latest version of
OpenGL. It is widely supported by mobile devices. OpenGL ES 3.0 is a
cross-platform API, so we can use it to develop games for both Android
and iOS.

3D models are stored in the Wavefront OBJ format. We will implement a
simple OBJ loader to load 3D models. The OBJ format is very simple, so
it is easy to implement.
\end{document}